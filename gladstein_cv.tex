%%%%%%%%%%%%%%%%%%%%%%%%%%%%%%%%%%%%%%%%%
% Medium Length Professional CV
% LaTeX Template
% Version 2.0 (8/5/13)
%
% This template has been downloaded from:
% http://www.LaTeXTemplates.com
%
% Original author:
% Trey Hunner (http://www.treyhunner.com/)
%
% Important note:
% This template requires the resume.cls file to be in the same directory as the
% .tex file. The resume.cls file provides the resume style used for structuring the
% document.
%
%%%%%%%%%%%%%%%%%%%%%%%%%%%%%%%%%%%%%%%%%

%----------------------------------------------------------------------------------------
%	PACKAGES AND OTHER DOCUMENT CONFIGURATIONS
%----------------------------------------------------------------------------------------

\documentclass{resume} % Use the custom resume.cls style
\usepackage[dvipsnames]{xcolor}
\usepackage{hyperref}

\usepackage[left=0.75in,top=0.6in,right=0.75in,bottom=0.6in]{geometry} % Document margins
\newcommand{\tab}[1]{\hspace{.2667\textwidth}\rlap{#1}}
\newcommand{\itab}[1]{\hspace{0em}\rlap{#1}}
\renewcommand{\thefootnote}{\fnsymbol{footnote}}

\usepackage{enumitem}

\name{ARIELLA GLADSTEIN, PhD} % Your name
%\address{} % Your address
\address{algladstein@gmail.com} % Your phone number and email
% \address{\small \url{https://ariella-gladstein.netlify.app}  \\ \url{https://github.com/agladstein} \\ \url{www.linkedin.com/in/ariella-gladstein}} % Your phone number and email
\address{\small \url{https://github.com/agladstein} \\ \url{www.linkedin.com/in/ariella-gladstein}} % Your phone number and email
\renewenvironment{rSection}[1]{
\sectionskip
\textcolor{RoyalPurple}{\MakeUppercase{#1}}
\sectionlineskip
\hrule
\begin{list}{}{
\setlength{\leftmargin}{1.5em}
}
\item[]
}{
\end{list}
}

\begin{document}
%----------------------------------------------------------------------------------------
%	EDUCATION SECTION
%----------------------------------------------------------------------------------------

\textcolor{RoyalPurple}{\bf Expertise:}
\textcolor{RoyalPurple}{\em population genetics, genomics, computational biology, data science, machine learning, statistics}

\begin{rSection}{Education}

{\bf University of Arizona, Tucson, AZ} \hfill August 2018 \\ 
PhD in Ecology and Evolutionary Biology, Minor in Mathematics 

{\bf Beloit College, Beloit, WI} \hfill May 2011 \\ 
B.S. in Mathematical Biology \& Russian, Cum Laude \\
Departmental Honors: Mathematical Biology

\underline{Undergraduate Study Abroad}:\\
\hspace*{0.75cm} {Lomonosov Moscow State University, Moscow, Russia} \hfill Fall 2009 \\ 
\hspace*{0.75cm} {\em Biology Department}\\
\hspace*{0.75cm} {Russian State University for the Humanities, Moscow, Russia} \hfill Spring 2009 \\ 
\hspace*{0.75cm} {\em Russian Studies}

\end{rSection}

%----------------------------------------------------------------------------------------
%	WORK EXPERIENCE SECTION
%----------------------------------------------------------------------------------------

\begin{rSection}{Experience}

{\bf Open Source Contributor} \hfill  2024 - present\\ 
{\em Tskit community} \hfill {\em Remote}\\
Contributing to writing and reviewing documentation and tutorials for the population genomics tskit ecosytem.

{\bf Research Scientist, Computational Biology} \hfill  2021 - 2024\\ 
{\em Ancestry Science} \hfill {\em Embark Veterinary, Remote}\\
Developed dog breed ancestry and relative-matching products using AWS cloud infrastructure and state of the art computational genomics methods.
\begin{itemize}[noitemsep, topsep=0pt]
    \item Proposed and developed a machine learning approach for village dog classification, enhancing accuracy and scalability.
    \item Conducted literature reviews, tested, and reported recommendations on published methods for population clustering and relative inference.
    \item Collaborated with scientists and engineers to improve local ancestry inference, implementing algorithm improvements for uncertainty estimates.
    \item Played a key role in improving AWS pipeline automation from 40\% to over 95\% and scaling processes from hundreds of thousands to millions of samples.
    \item Collaborated with a cross-functional team on a science initiative projected to save \$7.5 million annually. 
    \item Served on the Diversity, Equity, Inclusion, and Belonging (DEIB) committee, co-authoring an Inclusive Norms document and contributing to various DEIB initiatives.
\end{itemize}

{\bf Postdoctoral Fellowship} \hfill  2018 - 2020\\ 
{\em Schrider Lab} \hfill {\em Department of Genetics, University of North Carolina, Chapel Hill}\\
Developed deep learning models for population genetics inference. Contributed to writing documentation and tutorials for the population genomics tskit software ecosystem. Actively participated in the PopSim Consortium, contributing to the development of the stdpopsim library.


{\bf Dissertation Research} \hfill  2011 - 2018\\ 
{\em Hammer Lab} \hfill {\em Ecology and Evolutionary Biology, University of Arizona}\\
Dissertation: Inference of recent demographic history of population isolates using genome-wide high density SNP arrays and whole genome sequences \\
Developed and applied computational methods for population genetics analysis, addressing ascertainment bias in identifying runs of homozygosity and demographic inference. Created and released software for high-throughput genomic simulations. Applied Approximate Bayesian Computation (ABC) to model the demographic history of Ashkenazi Jews.

\newpage

{\bf Functional Genomics Research Rotation} \hfill  Spring 2012\\ 
{\em Restifo Lab} \hfill {\em Neuroscience, University of Arizona}

{\bf Computational Genomics Research Rotation} \hfill  Fall 2011\\ 
{\em Kececioglu Lab} \hfill {\em Computer Science, University of Arizona}

{\bf Lab technician intern} \hfill  Sp, Fall 2009\\ 
{\em Laboratory of Population Genetics} \hfill {\em Russian Academy of Medical Sciences, Moscow, Russia}

\end{rSection}


%----------------------------------------------------------------------------------------
%	Teaching Experience
%----------------------------------------------------------------------------------------

\begin{rSection}{Teaching Experience}
    
    {\bf Introduction to using stdpopsim Workshop} \hfill 2020\\
    Designed and presented interactive workshop on using stdpopsim, 
    the standard \hspace*{\fill} {\em Virtual, PopSim Consortium} \\  
    library for reproducible, bug-free simulations of genetic datasets from published \\
    demographic histories.

    {\bf NSF Cyber Carpentry Workshop} \hfill 2019\\
    Assisted with 2-week workshop, where participants learned best \hspace*{\fill} {\em University of North Carolina, Chapel Hill} \\ 
    practices for data-intensive computing, cloud infrastructure, and\\ 
    deep learning through hands-on projects.

    {\bf Research Bazaar Workshop on R} \hfill 2018\\ 
    Assisted with hands-on intro R workshop. \hspace*{\fill} {\em University of Arizona}

    {\bf CyVerse Container Camp} \hfill 2018\\
    Assisted with 3-day hands-on workshop on using containerized workflows. \hspace*{\fill} {\em University of Arizona}

    {\bf Software Carpentry Workshop} \hfill 2017\\
    Assisted with Software Carpentry hands-on workshop on basic Unix, Python, and Git \hspace*{\fill} {\em University of Arizona}

    {\bf Tucson Women’s Hackathon workshop on Git} \hfill 2017\\ 
    Assisted with hands-on workshop on basic Git \hspace*{\fill} {\em University of Arizona}

    {\bf ECOL 320 Genetics} \hfill 2014 - 2016\\ 
    Graduate Teaching Assistant \hspace*{\fill} {\em University of Arizona}

    {\bf ECOL 182L Intro to Ecology and Evolutionary Biology Lab} \hfill 2015\\
    Graduate Teaching Assistant \hspace*{\fill} {\em University of Arizona}

    {\bf Population Genetics Module} \hfill 2013\\
    Developed and taught a 3-day module on population genetics to high schoolers. \hspace*{\fill} {\em Kino School}

    {\bf BIOL 247 Biometrics} \hfill 2011\\ 
    Teaching Assistant \hspace*{\fill} {\em Beloit College}

    {\bf RUSS 110, 115, 210, 215 Russian} \hfill 2009 - 2011 \\ 
    Tutor \hspace*{\fill} {\em Beloit College}

\end{rSection}

%----------------------------------------------------------------------------------------
%	Skills
%----------------------------------------------------------------------------------------

\begin{rSection}{Skills}

    \begin{tabular}{ @{} >{\bfseries}l @{\hspace{6ex}} l }
    Programming & Python, R, Bash, unit testing, profiling, multiprocessing, Git\\
    HPC/HTC \& Cloud & Linux/Unix, SLURM, PBS, AWS, SageMaker, JupyterHub, Kubernetes  \\
    Genomics & PLINK, Bedtools, VCFtools, phasing, imputation, haplotype detection, population \\ & structure, genome simulation\\
    Data Science & Numpy, Pandas, Scikit-learn, SciPy, Keras, Matplotlib, Seaborn, Plotly, ggplot \\
    Reproducibility & Workflow development (Pegasus, Makeflow, Snakemake), containers \\ & (Singularity, Docker), virtual environments, Jupyter, Knitr \\
    Documentation & \LaTeX, Markdown, reStructuredText, JupyterBook, Confluence \\
    Project Management & Agile framework (Scrum, Kanban), Jira, Trello, Github, Gitlab \\
    Mathematical Skills & Probability theory, statistical inference, linear algebra \\
    Language Skills & English (native speaker), Russian (fluent)
    \end{tabular}
    
\end{rSection}

%----------------------------------------------------------------------------------------
%	Publications
%----------------------------------------------------------------------------------------
\newpage
\begin{rSection}{Publications}

\item Lauterbur, M. E.* et al. 2023. Expanding the stdpopsim species catalog, and lessons learned for realistic genome simulations \textit{eLife} 12:RP84874. \url{https://doi.org/10.7554/eLife.84874.3}

\item Baumdicker, F.* et al. 2022. Efficient ancestry and mutation simulation with msprime 1.0. \textit{Genetics}. Volume 220, Issue 3. \url{https://doi.org/10.1093/genetics/iyab229}

\item Adrion, J. R.*, Cole, C. B.*, Dukler, N.*, Galloway, J. G.*, \textbf{Gladstein, A. L.}\footnote{First author}, Gower, G.*, Kyriazis, C.C.*, Ragsdale, A.P.*, Tsambos, G.*, et al. 2020. A community-maintained standard library of population genetic models. \textit{eLife}. 9:e54967. doi: \url{https://doi.org/10.1101/2019.12.20.885129}

\item Bernstein M.N.* et al. Jupyter notebook-based tools for building structured datasets from the Sequence Read Archive [version 2; peer review: 2 approved]. F1000Research 2020. 9:376. doi: \url{https://doi.org/10.12688/f1000research.23180.2}

\item \textbf{Gladstein A.L.}* and Hammer M.F. 2019. Substructured population growth in the Ashkenazi Jews inferred with Approximate Bayesian Computation. \textit{Molecular Biology and Evolution}. 36(6): 1162-1171. doi: \url{https://dx.doi.org/10.1093/molbev/msz047}

\item \textbf{Gladstein A.L.}* et al. 2018. SimPrily: A Python framework to simplify high-throughput genomic simulations. \textit{SoftwareX}, 7, 335-340. \url{https://doi.org/10.1016/j.softx.2018.09.003}

%\item Gladstein A.L., Watkins J., and Hammer, M.F. 2018. Reliability of runs of homozygosity detection from SNP arrays depends on algorithmic choices. (in preparation).

\item \textbf{Gladstein A.}* and Hammer M.F. 2016. Population Genetics of the Ashkenazim. In: \textit{eLS}. John Wiley \& Sons, Ltd: Chichester. pp. 1-8. \url{https://doi.org/10.1002/9780470015902.a0020818.pub2}

\item Behar D.* et al. 2013. No evidence from genome-wide data of a Khazar origin for the Ashkenazi Jews. \textit{Human Biology}. 85.6:859-900. \url{https://doi.org/10.3378/027.085.0604}

\item \textbf{Gladstein A.L.}* 2011. Split decomposition analysis groups Jewish populations together between European and Middle Eastern populations. \textit{The Beloit Biologist}. 30:29-36.

\end{rSection}

%----------------------------------------------------------------------------------------
%	Publications
%----------------------------------------------------------------------------------------

\begin{rSection}{Posters and Presentations}

\textbf{Invited Talks}
\item  Deep learning for demographic model choice. Human Evolutionary and Population Genomics Seminar. 2020. LANGEBIO, UGA-Cinvestav. Irapuato-Guanajuato, Mexico (virtual).

\item Inference of evolutionary history with Approximate Bayesian Computation. Open Science Grid All-Hands Meeting. 2018. Salt Lake City, Utah.

\item Code optimization for research scientists. Research Bazaar. 2018. Tucson, AZ.

\textbf{Posters}

\item \textbf{Gladstein A.L.}\footnote{Presenter}, et al. A Novel Machine Learning Method for Classification of Village Dogs.  Probabilistic Modeling in Genomics. 2023. Cold Spring Harbor, NY.

\item \textbf{Gladstein A.L.}\textsuperscript{\textdagger}, Schrider R.D.  Demographic model selection with deep learning.  Probabilistic Modeling in Genomics. 2019. Aussois, France.

\item \textbf{Gladstein A.L.}\textsuperscript{\textdagger}, Hammer M.F. Substructured population growth in the Ashkenazi Jews inferred with Approximate Bayesian Computation. UNC Women in Computing Research Symposium. 2019. Chapel Hill, NC.

\item \textbf{Gladstein A.L.}\textsuperscript{\textdagger}, Hammer M.F. Substructured population growth in the Ashkenazi Jews inferred with Approximate Bayesian Computation. Probabilistic Modeling in Genomics. 2018. Cold Spring Harbor, NY.

\item \textbf{Gladstein A.L.}\textsuperscript{\textdagger}, et al. Efficient pipeline for whole genome simulation and summary statistic calculation with flexible demographic models. Meeting of the American Society for Human Genetics. 2017. Orlando, FL.

\item \textbf{Gladstein A.L.}\textsuperscript{\textdagger}, et al. The effect of SNP array ascertainment bias on the distribution of runs of homozygosity lengths. Annual Meeting of the American Society for Human Genetics. 2015. Baltimore, MD. 

\end{rSection}

%----------------------------------------------------------------------------------------
%	Workshops
%----------------------------------------------------------------------------------------

\newpage
\begin{rSection}{Workshops and Hackathons}

    SMBE satellite meeting on Speciation Genomics, Tjarno, Sweden (3 days) \hfill 06/2019\\
    NCBI RNA-Seq in the Cloud hackathon, Chapel Hill, NC (3 days) \hfill 03/2019\\
    Cyber Carpentry, Chapel Hill, NC (2 weeks) \hfill 06/2018\\
    XSEDE HPC Workshop: Big Data, Tucson, AZ (2 days) \hfill 02/2018\\
    Open Science Grid User School, Madison, WI (1 week) \hfill 07/2017
    
\end{rSection}

%----------------------------------------------------------------------------------------
%	Awards, Scholarships, & Grants
%----------------------------------------------------------------------------------------

\begin{rSection}{Awards}

XSEDE Research allocation (250,000 CPU hrs) \hfill 2020\\
XSEDE Supplemental allocation (250,000 CPU hrs) \hfill 2018, 2020\\
Probabilistic Modeling in Genomics Grant (registration, meals, lodging) \hfill 2018, 2019\\
XSEDE Startup allocation (100,000 CPU hrs, 6,500 GPU hrs, 5.5 Tb storage) \hfill 2019\\
XSEDE Startup allocation (150,000 CPU hrs) \hfill 2017\\
Open Science Grid User School (travel, lodging, meals, cost of program) \hfill 2017\\
GPSC Travel Grant (\$761) \hfill 2015, 2016, 2017 \\
University of Arizona Galileo Circle Scholarship (\$1,000) \hfill 2015 \\
NIH Computational and Mathematical Modeling of Biological Systems Traineeship (\$71,064) \hfill 2013 - 2014\\
NSF Integrative Graduate Education and Research Traineeship in Genomics (\$97,083) \hfill 2011 - 2013\\
Society for Learning Unlimited Grant (\$2,000) \hfill 2009\\
Study Abroad Enhancement Grant, Beloit College (\$250) \hfill 2009\\
Beloit College Presidential Scholar (\$60,000) \hfill 2007 - 2011

\end{rSection}


%----------------------------------------------------------------------------------------
%	Project Management
%----------------------------------------------------------------------------------------

% \begin{rSection}{Project Management}

%     Recruited and managed a team of 5 post-graduate and undergraduate students, including computer  \hfill {\em 2017}\\ 
%     scientists, a software engineer, and a mathematician, working on code development and high throughput 
%     computing for bioinformatics, resulting in the publication Gladstein et al. 2018.
    
% \end{rSection}

%----------------------------------------------------------------------------------------
%	Community Service
%----------------------------------------------------------------------------------------

\begin{rSection}{Community Service}

    Judged Graduate \& Professional Student Council Travel Grants, Tucson, AZ \hfill 2012, 2016, 2017, 2018\\
    Mentored undergraduate interns in computer science, Tucson, AZ \hfill 2017\\    
    Judged Tucson Magnet High School Science Fair, Tucson, AZ \hfill 2012, 2015\\
    Judged EEB Undergraduate Research Poster Session, Tucson, AZ \hfill 2012\\
    Arizona Assurance Mentor \hfill 2012

    
    \end{rSection}

%----------------------------------------------------------------------------------------
%	Study Abroad
%----------------------------------------------------------------------------------------

% \begin{rSection}{Study Abroad}

% {\bf Lomonosov Moscow State University} \hfill Fall 2009 \\ 
% {\em Biology Department} \hfill {\em Moscow, Russia}

% {\bf Russian State University for the Humanities} \hfill Spring 2009 \\ 
% {\em Russian Studies} \hfill {\em Moscow, Russia}

% \end{rSection}
%----------------------------------------------------------------------------------------

%----------------------------------------------------------------------------------------
%	Other Activities
%----------------------------------------------------------------------------------------

\begin{rSection}{Other Activities}

{\bf Circus Arts} \hfill 2012 - present \\ 
Trained, performed, and taught aerial silks, aerial rope, lyra, flying trapeze, tightwire, handbalancing, contortion.

{\bf Figure Skating} \hfill 1998 - 2013, 2021 - present \\ 
Trained, performed, competed, and taught singles freestyle.

\end{rSection}
%----------------------------------------------------------------------------------------


\end{document}